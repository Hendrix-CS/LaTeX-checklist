% -*- compile-command: "pdflatex LaTeX-checklist.tex" -*-
\documentclass{article}

\usepackage{amsmath}
\usepackage{amssymb}
\usepackage[most]{tcolorbox}

\title{Hendrix \protect\LaTeX\ checklist for students}
\date{}

\newtcblisting{good}{colframe=green!50!black}
\newtcblisting{bad}{colframe=red!50!black}

\begin{document}

\maketitle
\def\arraystretch{1.5}

\LaTeX\ is the \emph{de facto} standard system for creating and
distributing high-quality documents in mathematics and computer
science, and you may be required to use it in several classes
throughout the Mathematics and Computer Science majors.

For creating \LaTeX\ documents, we recommend using XXX Overleaf, a
free online editor that even allows multiple people to easily
collaborate on the same document.  For learning about the basics of
\LaTeX, we recommend working through some of the exercises in the XXX
Bates LaTeX Manual.

This document is a simple checklist you can use to ensure you are
XXX.  It can be found at XXX.

\section*{Formatting}
\label{sec:formatting}

\begin{itemize}
\item Use \verb|``| (two back ticks) and \verb|''| (two single quotes)
  to create quotation marks:
\begin{good}
``like this!''
\end{good}
If you use normal double quote charaters, the result will be typeset
using two ``close quote'' characters.
\begin{bad}
"These quote marks look bad."
\end{bad}


\item Use a contrasting style like a typewriter font for code, names
  of functions, \emph{etc.}
  \begin{good}When we run the \texttt{BFS} function\end{good}
\item Use \verb|\emph{...}| to italicize things like Latin abbreviations
  (such as \emph{e.g.} and \emph{i.e.}) or words you are going to
  define.
\begin{good}
A \emph{snoozle} is a special kind of matrix which has all entries less than $1/2$.
\end{good}
\item There are four kinds of horizontal lines:
  \begin{itemize}
  \item A minus sign is created by using a hyphen inside math mode.
    \begin{good}$(-2) - 3$\end{good}
    \begin{bad}$x$ = -4  % A minus sign outside of math mode\end{bad}
  \item A single hyphen is used for hyphenated words.
    \begin{good}helter-skelter\end{good}
  \item A double hyphen creates an ``en-dash'', which is used for
    ranges.
    \begin{good}Pages 23--69 discuss the period 1952--1987.\end{good}
  \item A triple hyphen creates an ``em-dash'', which is used as
    punctuation to signal a break in thought.  The usual typesetting
    style (in the US, at least) is to leave no spaces on either side
    of the em-dash.
    \begin{good}If this were true---which it isn't---then\end{good}
  \end{itemize}
\item Use \verb|itemize| or \verb|enumerate| for lists.
\begin{good}
\begin{itemize}
  \item One thing
  \item Another thing
  \item Last thing
\end{itemize}
\end{good}
  Using \verb|enumerate| instead of \verb|itemize| causes the items to
  be sequentially numbered.

\item If you use a command in the middle of some text, by default it
  will eat any following whitespace:
\begin{bad}
\LaTeX is great
\end{bad}
  Follow the command with a backslash before the space to force an
  explicit space to appear, like this:
\begin{good}
\LaTeX\ is great
\end{good}
\item \LaTeX\ automatically puts extra space at the end of every
  sentence, but it thinks any punctuation is the end of a sentence, so
  abbreviations can have too much space after them.  Use a backslash
  before the space to force a normal space.  (But this is kind of nitpicky
  since it can be hard to even tell the difference!)
\begin{bad}
Dr. Yorgey likes \LaTeX
\end{bad}
\begin{good}
Dr.\ Yorgey likes \LaTeX
\end{good}

\end{itemize}

\section*{Math}
\label{sec:math}

\begin{itemize}
\item Use the \texttt{amsmath} and \texttt{amssymb} packages by
  putting \verb|\usepackage{amsmath}| and \verb|\usepackage{amssymb}|
  in the \emph{preamble} of your document (that is, the part before
  \verb|\begin{document}|).
\item Use math mode appropriately:
  \begin{itemize}
  \item Use dollar signs for inline mathematics (including variable
    names, numbers, equations, formulas, \dots)
    \begin{good}
In this case, $x$ cannot possibly be $-3$, because we assumed that $x > 10$.
    \end{good}
  \item Use \verb|\[ ... \]| for formulas or equations that should go
    on their own line.
    \begin{good}
Solving for $x$, we find that \[ x = 2 + y^2 - \pi, \] which is what we wanted to show.
    \end{good}
 \item Use \verb|$ ... $| or \verb|\[ ... \]| for an entire
   equation/formula, instead of building it out of lots of little
   pieces. Use \verb|\text{...}| if you need some normal text in the
     middle of your formula.
   \begin{bad}
$x$ = $\pm 2$ or $(\sqrt{7} - \pi)$
   \end{bad}
   \begin{good}
$x = \pm 2 \text{ or } (\sqrt{7} - \pi)$
   \end{good}
  \end{itemize}
\item Use \verb|\mathit{...}| for longer variable names in math mode.
  \begin{bad}
$myFunction(x) \times 2^n$
  \end{bad}
  \begin{good}
$\mathit{myFunction}(x) \times 2^n$
  \end{good}
\item Use commands like \verb|\log|, \verb|\sin|, \verb|\cos|,
  \verb|\lim| appropriately.
  \begin{bad}
$n log(n) + lim_{x \to \infty} \frac{cos \theta}{x^2}$
  \end{bad}
  \begin{good}
$n \log(n) + \lim_{x \to \infty} \frac{\cos \theta}{x^2}$
  \end{good}
\item Use \verb|\dots| for ellipses and \verb|\cdots| for centered ellipses.
  \begin{bad}
$\{1, 2, ..., 10\}$
  \end{bad}
  \begin{bad}
$1 + 2 + ... + 10$
  \end{bad}
  \begin{good}
$\{1, 2, \dots, 10\}$
  \end{good}
  \begin{good}
$1 + 2 + \cdots + 10$
  \end{good}
\item Use \verb|\begin{align} ... \end{align}| for multi-step
  derivations.
\item Use \verb|\mathbb{...}| for appropriate symbols.
  \begin{itemize}
  \item Example: \verb|$\mathbb{R} - \mathbb{N}$| looks like: $\mathbb{R} - \mathbb{N}$
  \end{itemize}
\item Use \verb|{...}| around multi-character exponents, subscripts, \emph{etc.}
  \begin{itemize}
  \item Example (bad): \verb|$2^15$| looks like: $2^15$
  \item Example: \verb|$2^{15}$| looks like: $2^{15}$
  \end{itemize}
\item Use \verb|\left| and \verb|\right| to make sure delimiters are
  big enough.
  \begin{itemize}
  \item Example (bad): %
    \verb|$(\frac{\pi}{\sum_{k=1}^{\infty} \frac{1}{k^2}})$| looks
    like $(\frac{\pi}{\sum_{k=1}^{\infty} \frac{1}{k^2}})$
  \item Example: %
    \verb|$\left( \frac{\pi}{\sum_{k=1}^{\infty} \frac{1}{k^2}} \right)$| looks
    like $\left(\frac{\pi}{\sum_{k=1}^{\infty} \frac{1}{k^2}}\right)$
  \end{itemize}
\end{itemize}

\section*{Organization}

\begin{itemize}
\item Use \verb|\chapter|, \verb|\section|, \verb|subsection|,
  \emph{etc.} appropriately
\item Use \verb|\label| and \verb|\ref| appropriately to refer to
  numbered sections, theorems, figures, \emph{etc.}
\end{itemize}


\end{document}
