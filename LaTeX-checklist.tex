% -*- compile-command: "pdflatex LaTeX-checklist.tex" -*-
\documentclass{article}

\title{\protect\LaTeX\ checklist}

\begin{document}

\maketitle

Blah blah

\section*{Formatting}
\label{sec:formatting}

\begin{itemize}
\item Use \verb|``| (two back ticks) and \verb|''| (two single quotes)
  to create quotation marks, \verb|``like this''|.  The result should
  look ``like this''.  If you use normal double quote charaters, the
  result will look bad, "like this".
\item Use a contrasting style like a typewriter font for code, names
  of functions, \emph{etc.}
  \begin{itemize}
  \item Example: \verb|When we run the \texttt{BFS} function,|
  \item Looks like this: When we run the \texttt{BFS} function,
  \end{itemize}
\item Use \verb|\emph{...}| to italicize things like Latin abbreviations
  (\emph{e.g.}, \emph{e.g.} and \emph{i.e.}) or words you are going to
  define (``A \emph{snoozle} is a special kind of matrix which\dots'').
\item There are four kinds of horizontal lines:
  \begin{itemize}
  \item A minus sign is created by using a hyphen inside math mode.
    \begin{itemize}
    \item Example: \verb|$-3$| looks like: $-3$
    \item Example (bad): \verb|-$3$| looks like: -$3$
    \end{itemize}
  \item A single hyphen is used for hyphenated words.
    \begin{itemize}
    \item Example: \verb|helter-skelter| looks like: helter-skelter
    \end{itemize}
  \item A double hyphen creates an ``en-dash'', which is used for
    ranges.
    \begin{itemize}
    \item \verb|Please read pages 23--69| looks like: Please read
      pages 23--69
    \end{itemize}
  \item A triple hyphen creates an ``em-dash'', which is used as
    punctuation to signal a break in thought
    \begin{itemize}
    \item Example: \verb|If this were true---which it isn't---then|
      looks like: If this were true---which it isn't---then
    \end{itemize}
  \end{itemize}
\item Use \verb|itemize| or \verb|enumerate| for lists. For example,
\begin{verbatim}
\begin{itemize}
  \item One thing
  \item Another thing
  \item Last thing
\end{itemize}
\end{verbatim}
  looks like this:

  \begin{itemize}
  \item One thing
  \item Another thing
  \item Last thing
  \end{itemize}

  Using \verb|enumerate| instead of \verb|itemize| causes the items to
  be sequentially numbered.

\item If you use a command in the middle of some text, by default it
  will eat any following whitespace; for example,%
  \verb|\LaTeX is great| is typeset like this: \LaTeX is great.
  Follow the command with a backslash before the space to force an
  explicit space to appear, like this: \verb|\LaTeX\ is great|.
\item \LaTeX\ automatically puts extra space at the end of every
  sentence, but it thinks any punctuation is the end of a sentence, so
  abbreviations can have too much space after them.  Use a backslash
  before the space to force a normal space.  For example:
  \begin{itemize}
  \item \verb|Dr. Yorgey| looks like this (bad): Dr. Yorgey
  \item \verb|Dr.\ Yorgey| looks like this (good): Dr.\ Yorgey
  \end{itemize}
\end{itemize}



\end{document}
